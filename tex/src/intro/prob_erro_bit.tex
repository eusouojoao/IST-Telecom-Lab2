%
%=============================--D--=============================%
\clearpage
\subsection{4. Probabilidade de erro do bit}
%A ideia do projeto é criarmos um emissor (modulador) que envia um conjunto de bits e sermos capazes com o nosso recetor descodificar a mensagem enviada. 
%Idealmente, na ausência de ruído branco, a sequência de bits descodificados no recetor seria exatamente iguais à transmitida, mas numa situação real existem sempre bits que são erroneamente detetados no seio do ruído.%

Antes de proceder para as etapas de tratamento de dados subsequentes é fundamental analisar a métrica "probabilidade de erro de bit", $P_e$, com vista a fornecer uma preâmbulo à exposição e análise de resultados. Invocando o Teorema da Probabilidade Total e supondo a equiprobabilidade dos bits `1' e `0', a probabilidade de erro será dada por:

\begin{equation*}
    P_e= P(\text{erro}\ |\ \text{bit 1}) P(\text{bit 1})\ +\ P(\text{erro}\ |\  \text{bit 0})P(\text{bit 0})\
\end{equation*}
\[P_e =\dfrac{P(\text{erro}\ |\ \text{bit 1})\ +\ P(\text{erro}\ |\ \text{bit 0})}{2}\]
%
%=============================--D--=============================%
\subsubsection{4.1 Probabilidade de erro  de bit para um sinal binário polar}
\label{subsubsec:prob-erro}
Reconhecendo que o sinal binário polar\footnotemark[10] (onde se encontram os sinais das modulações em análise) pode ser expresso procedimentalmente sobre a forma $y(t)=\sum_{n=-\infty}^{\infty} a_n p(t-nT)$ de amplitudes $a_n=\pm \text{A}$, após a sincronização de símbolo e encontro do instante de amostragem ideal $T_0$ temos:

\begin{equation*}
    p_k=a_n \cdot p_0=   
    \begin{cases}
        +\text{A} \cdot p(T_0)\text{,} & \text{bit 1}\\
        -\text{A} \cdot p(T_0)\text{,} & \text{bit 0} 
    \end{cases}
\end{equation*}

\begin{center}
     em que $p(t)$ denomina a resposta do sistema transmissão-receção (vide a  \hyperref[subsec:filtro-adaptado]{secção 2.}).\\
\end{center}


Podemos então determinar as probabilidades de erro condicionadas:
\vskip -1em
$$
\begin{tabular}{c}
$
    \begin{cases}
        P(\text{erro}\ |\ \text{bit 1}) = P(\text{erro}\ |\ a_n=+A)=P(n_k < -A\cdot p(T_0))\\
        P(\text{erro}\ |\ \text{bit 0})=P(\text{erro}\ |\ a_n=-A)=P(n_k > +A\cdot p(T_0))
    \end{cases}
$
\end{tabular}
\mkern-18mu\mkern-9mu \rightarrow Q\left( \dfrac{A \cdot p(T_0)}{\sigma_n} \right)
$$
E subsequentemente concluir que a probabilidade erro de um sinal binário polar é:

\begin{equation*}\label{eq:6}
    P_e=Q\left( \dfrac{A\cdot p(T_0)}{\sigma_n} \right)
\end{equation*}

%\begin{center}
em que $\sigma_n$ é o desvio padrão do ruído no instante de amostragem.   
%\end{center}
Reescrevendo a probabilidade em função da energia de bit obtemos:

\begin{equation*}\label{eq:7}
    P_e=Q\left( \sqrt{\frac{2 E_b}{N_0}} \right)
\end{equation*}
que é válida para ambas as modulações em análise\footnotemark[11].
\vskip 1em
\noindent\fcolorbox{black}{white}{%
        \minipage[t]{\dimexpr\linewidth-2\fboxsep-2\fboxrule\relax}
            \textbf{Nota} $\rightarrow$ Embora coerente para a probabilidade de erro de bit, o mesmo não acontece para a probabilidade de erro de símbolo: $P_\text{es} \approx \text{log}_2(M) P_e$, admitindo um M = 4 para QPSK e M = 2 para BPSK (\hyperref[subsec:intro]{M-PSK}), observamos que:
            \vskip -1em
            $$
                P_{es\_\text{QPSK}} \approx 2P_{es\_\text{BPSK}}
            $$
        \endminipage}

\footnotetext[10]{Sinal à saída do filtro adaptado, vide \hyperref[subsec:filtro-adaptado]{secção 2}.}
\footnotetext[11]{"QPSK can be regarded as a pair of orthogonal BPSK systems (...)  Because they are orthogonal, they don't interfere (to a good approximation), hence the BER curves are largely equivalent."\cite{dsp}}